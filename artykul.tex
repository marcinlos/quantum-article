\documentclass[10pt]{article}

\usepackage{csagh}

\usepackage[utf8]{inputenc}
\usepackage{lmodern} 
\usepackage[T1]{fontenc}
\usepackage{microtype}

%\usepackage[pdftex]{graphicx}
%\usepackage[pdftex]{hyperref} 
%\usepackage{bookmark}
\usepackage{url}

\usepackage{polski}
\usepackage[polish]{babel}
\selectlanguage{polish}


\begin{document}
\begin{opening}

\title{Quantum Commitment}
\author[AGH University of Science and Technology, los@student.agh.edu.pl]{Marcin Łoś}

\begin{abstract}
  W sumie nic takiego, parę artykułów pocytowane i tyle.
\end{abstract}

\keywords{quantum commitment}

\end{opening}

\section{What is a commitment scheme?}

A commitment scheme is a cryptographic protocol that allows one (henceforth known as ,,Alice'') to make a
permanent choice (\emph{commitment}), and provide a prove of this fact to other parties (henceforth known
as ,,Bob''), without immediately revealing the choosen value, while at the same time making it possible 
for Alice to reveal it later in such a way that Bob can verify it is indeed the value Alice has choosen 
initially. Hence, there are two basic goals of a commitment scheme:

\begin{itemize}
  \item \emph{concealing} -- the proof must not provide Bob with information sufficient to determine the
    choosen value. This is necessary to ensure safety of Alice -- secrecy of her choice.
  \item \emph{binding} -- the proof must provide Bob with information sufficient to check if a particular
    value is the value choosen by Alice. This is necessary to ensure safety of Bob -- simply speaking,
    Alice must not be able to ,,cheat'', i.e. to make Bob believe she has choosen a value different than
    the one she actually has.
\end{itemize}

Intuitively, there seems to be a tradeoff between these two properties -- on the one hand, empty proof
is perfectly concealing, but does not bind Alice in any way, and revealing the full value is perfectly
binding, while obviously not concealing.


Usually, commitment schemes consist of two phases:

\begin{itemize}
  \item \emph{commitment phase} -- Alice chooses the value, and produces a proof for Bob
  \item \emph{reveal phase} -- Alice reveals choosen value to Bob, who verifies it using his proof
\end{itemize}

As it turns out, commitment schemes constitute a rich and important branch of cryptography, with lots of
applications both practical and theoretical, as building blocks for other protocols. One simple example,
motivating the need for such device, and showing naturally arising problems it solves, is a game of 
rock-paper-scissor. Two people form one of three shapes with their palm, and simultaneously display it
to one another. Their choices determine the outcome -- one of them wins, or there is a draw. In the
language of commitment schemes, used above, two people commit to one of 3 possible values, and then reveal
their choice. It is vital that neither knows the choice of the other before he makes his own, otherwise
he could always choose a winning shape, and inevitably win every time, rendering the game pointless.
This illustrates the need for concealing -- the choice must be secret. In practice, this is ensured
by forming the shape keeping the hand behind one's back. Furthermore, it must not be possible to change
the value after commitment. Otherwise, after one of the players reveals his choice, the other could again
switch to winning shape. In practice, players reveals their values simultaneously, thus in principle it
should not be possible to cheat.

Of course, someone with superhuman reflexes and perception could still manage to switch sufficiently 
fast -- it's an inherent theoretical security flow. With a proper commitment scheme, there would be no
such problem, as long as both players commit their choice before reveal phase.


\section{Commitment schemes in modern cryptography}

In this section, we shall explore briefly some of the applications of commitment schemes in various
areas of cryptography, to provide further motivation and emphasize importance of the subject.

\subsection*{Zero-knowledge proofs}

Zero-knowledge proof (protocol) is a method of proving some statement without revealing anything more
than what is required to verify the proof. The problem bears resemblance to the commitment problem, but
is more general. Nice introduction to the subject, with motivating examples and precise definitions can
be found in the first two chapters of \cite{MIT:StatZero}. 

Common special case is proving that one possesses some information without revealing it.
One simple example comes from asymmetric cryptography: if Alice posesses the private key matching Bob's
public key, she can prove it to Bob in the following manner: Bob chooses a random string, encrypts it
with the public key, and sends it to Alice, who subseqnently uses her private key to decrypt the message,
and sends it to Bob in plaintext. Assuming security of choosen encryption algorithm, Bob can conclude 
with high probability that if the plaintext matches his original encrypted message, Alice does indeed
possess the private key. Furthermore, if Alice does not have the private key, she has no way to obtain
the plaintext, short of breaking the encryption.

Commitment schemes often prove useful as building blocks of zero-knowledge protocols. Well-known example
is a way for Alice to prove she knows the Hamiltonian cycle in certain graph, without revealing the
cycle. Detailed discussion and analysis can be found in the first section of \cite{CM:CryptoNotes}. 
Let \(G\) be the graph. The protocol consists of following steps:

\begin{enumerate}
  \item Alice chooses some graph \(G'\) isomorphic to \(G\) and isomorphism \(\sigma\colon G\cong G'\)
  \item Alice commits to her choice
  \item Bob can now ask Alice for one of two things:
    \begin{enumerate}
      \item graph \(G'\) and isomorphism \(\sigma\)
      \item image of Alice's hamiltonian cycle by \(sigma\), i.e. hamiltonian cycle in \(G'\)
    \end{enumerate}
\end{enumerate}

If Alice can consistently produce acceptable responses to Bob's questions, he can conclude with high
probability that she has indeed found the hamiltonian cycle in \(G\). Otherwise, she would either fail
to exhibit isomorphism between \(G\) and her choosen graph \(G'\), even if she knew a hamiltonian
cycle in \(G'\), or she wouldn't be able to find hamiltonian cycle in \(G'\), even if she knew an
isomorphism \(G\cong G\). At the same time, Bob learns nothing about the cycle in \(G\), since he
either gets the isomorphism and no cycle, or he gets cycle in some graph \(G'\neq G\), but without
full knowledge of \(G'\) and the isomorphism he cannot pull it back to \(G\) (even with full knowledge
of \(G'\) he cannot easily find an isomorphism by himself, as no polynomial algorithm for Graph 
Isomorphism Problem is known).


\nocite{*}

\bibliographystyle{cs-agh}
\bibliography{bibliography}

\end{document}
